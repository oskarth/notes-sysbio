\documentclass[12pt]{report}

\usepackage{hyperref}

\newcommand{\link}[2]{\href{#1}{#2}}


\begin{document}

\title{Notes on System Biology}
\author{Oskar Thor{\'e}n\thanks{Email: me@oskarth.com}}

\maketitle

\chapter{Introduction}

To me it seems like biology is in a similar position to chemistry and
physics back in the day - it's getting a lot more rigorous and there's
plenty of things to discover. Especially with the application of
mathematical and computational tools.

The main question I'm interested in right now is: \textbf{how do cells
  make decisions}? Cells are simple enough to study but complex enough
to make it interesting. They commit suicide for the greater good
(apoptosis), they specialize (cell differentiation), and they divide
into two (the cell cycle). All of these areas relate to a notion from
dynamics (originated in celestial mechanics) - bistability.

In order to make the study rigorous, there are various mathematical
tools that have to be understood and explained. In particular, these
are tools such as numerical analysis / simulation, bifurcation
analysis, and perturbation methods. All of these tools allow us to
understand what is happening, why, and under what conditions. Even
though these notes are about system biology, the majority of it will
be taken up by the investigation of these mathematical tools.

\chapter{Papers}

\section{Verdugo et al. (2013)}

\link{http://rsob.royalsocietypublishing.org/content/royopenbio/3/3/120179.full.pdf}{link}

The main paper of interest.

\chapter{Perturbation theory}

\section{Introduction}

Perturbation theory is useful whenever you can't find exact solutions
to a system of differential equations. In essence it's about dividing
a problem into a solvable part and an approximated part.

The main text source we will use is \textit{A first look at
  perturbation theory} by Simmonds. We will also look at \textit{Lin
  and Segel} for an application.

\end{document}
